% -*- coding: utf-8 -*-


\chapter{The Tikz Package}


The {\scshape pdf}\ package, where ``{\scshape pdf}'' is supposed to mean ``portable
graphics format'' (or ``pretty, good, functional'' if you
prefer\dots), is a package for creating graphics in an ``inline''
manner. It defines a number of \TeX\ commands that draw
graphics. For example, the code \verb|\tikz \draw (0pt,0pt) -- (20pt,6pt);|
yields the line \tikz \draw (0pt,0pt) -- (20pt,6pt); and the code \verb|\tikz \fill[orange] (1ex,1ex) circle (1ex);| yields \tikz
\fill[orange] (1ex,1ex) circle (1ex);.

In a sense, when you use {\scshape pdf}\ you ``program'' your graphics, just
as you ``program'' your document when you use \TeX.  You get all
the advantages of the ``\TeX-approach to typesetting'' for your
graphics: quick creation of simple graphics, precise positioning, the
use of macros, often superior typography. You also inherit all the
disadvantages: steep learning curve, no \textsc{wysiwyg}, small
changes require a long recompilation time, and the code does not
really ``show'' how things will look like.



